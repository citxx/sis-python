\documentclass[a4paper, 12pt]{article}

% languages
\usepackage{ucs}
\usepackage[utf8x]{inputenc}
\usepackage[T2A]{fontenc}
\usepackage[russian]{babel}
\usepackage{upquote}

% spacing
\usepackage{geometry}
\geometry{top = 2cm}
\geometry{bottom = 2cm}
\geometry{left = 2cm}
\geometry{right = 2cm}

\usepackage{xcolor}
\usepackage{listings}
\lstset{inputencoding=utf8x, extendedchars=false}
\lstdefinestyle{python}{
	belowcaptionskip=1\baselineskip,
	breaklines=true,
	breakatwhitespace=true,
	frame=single,
	frameround=ffff,
	xleftmargin=\parindent,
	language=Python,
	showstringspaces=false,
	basicstyle=\ttfamily,
	keywordstyle=\bfseries\color{blue!60!black},
	commentstyle=\itshape\color{green!60!black},
	identifierstyle=\color{blue!10!black},
	stringstyle=\color{orange},
	rulecolor=\color{white!85!black},
	backgroundcolor=\color{white!97!black}
}
\lstset{style=python}

\pagestyle{empty}

\renewcommand{\labelitemi}{-}
%\renewcommand{\labelitemii}{\textopenbullet}

\begin{document}

\paragraph{Форматирование}

\begin{itemize}
\item Используйте 4 пробела для отступов.
\item Никогда не смешивайте пробелы и символы табуляции.
\item Не делайте строки длиннее 79 символов. Разделяйте строки на несколько с помощью заключения выражения в скобки. Этот способ лучше, чем использование обратного слэша. Делайте соответствующий отступ при переносе. При разбиении строки около бинарного оператора делайте разрыв \textbf{после} него.
\begin{lstlisting}[texcl]
var = (a * 10 +
       b / 15)    # Правильно
var = a * 10 \
    + b / 15      # Неверно
\end{lstlisting}

\item Пишите \lstinline{import} каждого модуля в отдельной строке
\item Располагайте все import'ы в верхней части файла перед любыми глобальными объявлениями.

\item Не используйте лишних пробелов в следующих ситуациях:
    \begin{itemize}
    \item Сразу после скобок.
        \begin{lstlisting}[texcl]
spam(ham[1], {eggs: 2})        # Правильно
spam( ham[ 1 ], { eggs: 2 } )  # Неверно
		\end{lstlisting}
    \item Перед запятой, точкой с запятой, двоеточием.
        \begin{lstlisting}[texcl]
if x == 4: print x, y; x, y = y, x       # Правильно
if x == 4 : print x , y ; x , y = y , x  # Неверно
		\end{lstlisting}
    \item Перед открывающей круглой скобкой при вызове функции и перед открывающей квадратной скобкой, обозначающей индексирование или срез.
        \begin{lstlisting}[texcl]
spam(kind['key'], lst[1:3])     # Правильно
spam (kind ['key'], lst [1:3])  # Неверно
		\end{lstlisting}
    \end{itemize}

\item Всегда окружайте следующие бинарные операторы ровно одним символом пробела с каждой стороны:
    \begin{itemize}
    \item присваивания (\lstinline{=}, \lstinline{+=}, \lstinline{-=} и т.д.),
    \item сравнения (\lstinline{==}, \lstinline{<}, \lstinline{>}, \lstinline{!=},
		\lstinline{<>}, \lstinline{<=}, \lstinline{>=}, \lstinline{in}, \lstinline{not in},
		\lstinline{is}, \lstinline{is not}),
    \item логические (\lstinline{and}, \lstinline{or}, \lstinline{not}),
    \item арифметические (\lstinline{+}, \lstinline{-}, \lstinline{*}, \lstinline{/}, \lstinline{//}).
    \end{itemize}

\item Не располагайте несколько инструкций в одной строке. Разнесите их по разным строкам.
    \begin{lstlisting}[texcl]
x = 3            # Правильно
func(10)
x = 3; func(10)  # Неверно 
	\end{lstlisting}
\item Не располагайте блок из нескольких инструкций на той же строке сразу после двоеточия (после \lstinline{if}, \lstinline{while} и т.д.).

\end{itemize}

\paragraph{Комментарии}

\begin{itemize}
\item Комментарии, противоречащие коду, хуже, чем их отсутствие.
\item Располагайте однострочные комментарии после кода в той же строке и отделяйте их от кода не менее, чем двумя пробелами. Комментарии должны начинаться с \lstinline{#} и одного пробела.
\end{itemize}

\paragraph{Имена}

\begin{itemize}
	\item Не используйте символы '\lstinline{l}', '\lstinline{O}', и '\lstinline{I}' как имена переменных. В некоторых шрифтах они могут быть очень похожи на цифры.

	\item Имена переменных и функций должны содержать только маленькие буквы. Слова разделяются символами подчёркивания. Примеры:
	\begin{lstlisting}
name, name_from_several_words
	\end{lstlisting}

	\item Имена констант должны содержать только заглавные буквы. Слова разделяются символами подчёркивания. Примеры:
	\begin{lstlisting}
NAME, NAME_FROM_SEVERAL_WORDS
	\end{lstlisting}
\end{itemize}

\paragraph{Программирование}

\begin{itemize}
	\item Сравнение с такими значениями как \lstinline{None} должно осуществляться с помощью операторов \lstinline{is} и \lstinline{is not}, а не операторами сравнения.
	\item Используйте методы \lstinline{''.startswith()} и \lstinline{''.endswith()} вместо срезов для проверки начала и конца строк.
	\item Для последовательностей (строк, списков, кортежей) используйте факт, что пустая последовательность эквивалентна \lstinline{False}.
	\item Не проверяйте переменные логического типа на \lstinline{True} и \lstinline{False} используя ==.
\end{itemize}

\paragraph{Функции}

\begin{itemize}
	\item Не ставьте пробелы вокруг знака \lstinline{=} при передаче функции именованных аргументов.
	\begin{lstlisting}[texcl]
f(5, x=7)    # Правильно
f(5, x = 7)  # Неверно
	\end{lstlisting}

	\item Разделяйте определения функций двумя пустыми строками.

	\item Разделяйте тело функции на логические части одной пустой строкой.
\end{itemize}

\paragraph{Исключения из правил} \

Перечисленные выше правила имеет смысл нарушать в тех случаях, когда следование им понижает читаемость кода даже для людей, привыкших к этим правилам. Трижды подумайте, прежде чем делать это.

\end{document}
