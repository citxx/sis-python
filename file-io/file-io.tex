\documentclass[a4paper, 12pt]{article}

% languages
\usepackage{ucs}
\usepackage[utf8x]{inputenc}
\usepackage[T2A]{fontenc}
\usepackage[russian]{babel}
\usepackage{upquote}

% spacing
\usepackage[top=2cm,bottom=2cm,left=2cm,right=2cm]{geometry}

\usepackage{xcolor}
\usepackage{listings}
\lstset{inputencoding=utf8x, extendedchars=false}
\lstdefinestyle{python}{
    belowcaptionskip=1\baselineskip,
    breaklines=true,
    breakatwhitespace=true,
    frame=single,
    frameround=ffff,
    xleftmargin=\parindent,
    language=Python,
    showstringspaces=false,
    basicstyle=\ttfamily,
    keywordstyle=\bfseries\color{blue!60!black},
    commentstyle=\itshape\color{green!60!black},
    identifierstyle=\color{blue!10!black},
    stringstyle=\color{orange},
    rulecolor=\color{white!85!black},
    backgroundcolor=\color{white!97!black}
}
\lstset{style=python}

\pagestyle{empty}

\renewcommand{\labelitemi}{-}
%\renewcommand{\labelitemii}{\textopenbullet}

\begin{document}

\section*{Файловый ввод-вывод в python}

Для открытия файла в python, как на чтение, так и на запись, используется встроенная функция \lstinline{open}, которая возвращает открытый файл. Например:
\begin{lstlisting}[texcl]
file_r = open('task.in', 'r')   # Открытие файла на чтение
file_w = open('task.out', 'w')  # Открытие файла на запись
\end{lstlisting}

Первый аргумент функции \lstinline{open}~--- имя файла, который нужно открыть. Второй аргумент~--- режим открытия файла (\lstinline{'r'}~--- для чтения, \lstinline{'w'}~--- для записи).

Теперь переменные \lstinline{file_r} и \lstinline{file_w} можно использовать для чтения из файла и для записи в файл соответственно:

\begin{lstlisting}[texcl]
line = file_r.readline()  # Читает и возвращает очередную строку
                          # из файла. Аналогична функции input().

content = file_r.read()  # Читает весь файл и возвращает его
                         # содержимое как строку. Например,
                         # при следующем файле:
                         # 1
                         # abc
                         # 5
                         # результат будет: '1\textbackslash nabc\textbackslash n5\textbackslash n'.

lines = file_r.readlines()  # Читает весь файл и возвращает
                            # его содержимое, как список строк.

file_w.write('any string')  # В качестве единственного аргумента
                            # принимает строку и печатает её в файл
                            # без последующего перевода строки.

print(1, '345', file=file_w)  # Функция print имеет именованный
                              # аргумент file, позволяющий выводить
                              # данные в файл вместо экрана.

file_r.close()   # Функция close используется для закрытия файла
file_w.close()
\end{lstlisting}

В некоторых случаях удобно использовать следующий способ чтения файла:

\begin{lstlisting}[texcl]
for line in file_r.readlines():  # Цикл по всем строкам файла
    print(line, file=file_w)
\end{lstlisting}

Этот пример выведет в файл \lstinline{file_w} содержимое файла \lstinline{file_r}. Вместо второй строки (с функцией \lstinline{print}) вы можете написать любой свой код.
\vfill

\hrulefill

{\footnotesize Примечания:
\begin{itemize}
\item символ \lstinline{#} используется для обозначения начала комментария до конца строки. Все символы после него будут проигнорированы интерпретатором python.
\item сочетание обратного слэша и символа \lstinline{'n'} (\lstinline{\n}) используется в строках для обозначения символа перевода строки.
\end{itemize}
}
\end{document}

